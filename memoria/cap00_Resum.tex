% ---------------------------------------------------------------------
% ---------------------------------------------------------------------
% ---------------------------------------------------------------------

\section*{Resum}
%Entre 50 i 200 paraules.
El modelat acústic és una tasca de processament del llenguatge natural molt activa en inte\lgem igència artificial, particularment per a reconeixement automàtic de la parla.
Recentment, aquesta tasca ha rebut gran atenció per part de grans companyies tecnològiques gràcies a les millores de rendiment obtingudes mitjançant la incorporació de tècniques avançades d'aprenentatge automàtic.
Un dels principals motius que explica aquesta gran atenció és l'enorme creixement de plataformes de difusió de continguts audiovisuals en ``streaming'' i videoconferència.
En aquest context, un aspecte molt important del modelat acústic és la seua integració en sistemes de reconeixement de la parla en directe, ja que no es pot aplicar de la mateixa manera a com es fa en el cas de la parla en diferit.
En aquest treball es proposa estudiar i implementar sistemes avançats de modelat acústic per a reconeixement automàtic de la parla en directe.
El resultant model acústic, combinat amb un model de llenguatge desenvolupat per investigadors del grup MLLP-VRAIN, donarà lloc a un sistema ASR híbrid que donarà servei al Conseil Européen pour la Recherche Nucléaire (CERN).
Per tal de dur a terme el treball, es farà ús de dades, tecnologia i experiència del grup MLLP del VRAIN, adquirits en el marc de projectes de recerca i transferència tecnològica desenvolupats en els últims anys.

%Màxim 5 paraules clau. 
\textbf{Paraules clau}: Reconeixement Automàtic de la Parla; Intel\lgem igència Artificial; Modelat Acústic; Aprenentatge Automàtic; Streaming.

\medskip
\rule{\textwidth}{0.5pt}

% ---------------------------------------------------------------------
% ---------------------------------------------------------------------
% ---------------------------------------------------------------------

\section*{\textit{Abstract}}

\begin{ParrafoIngles}
%Between 50 and 200 words.
Acoustic modeling is a very active natural language processing task in artificial intelligence, particularly for automatic speech recognition.
Recently, this task has received much attention from major technology companies due to the performance improvements obtained by incorporating advanced machine learning techniques.
One of the main reasons for this attention is the huge growth of streaming platforms for audiovisual content and videoconferencing. In this context, a very important aspect of acoustic modeling is its integration into live speech recognition systems, as it cannot be performed in the same way as in the case of offline speech recognition. In this work it is proposed to study and implement advanced acoustic modeling systems for automatic recognition of live speech. 
The resulting acoustic model, combined with a language model developed by researchers from the MLLP-VRAIN group, will result in a hybrid ASR system serving CERN.
In order to carry out the work, data, technology and experience from the VRAIN MLLP group will be used, acquired within the framework of research and technological transfer projects developed in recent years.

\textbf{Keywords}: Automatic Speech Recognition; Artificial Intelligence; Acoustic Model, Machine Learning; Streaming.

\end{ParrafoIngles}
% ---------------------------------------------------------------------

