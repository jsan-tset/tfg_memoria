% ---------------------------------------------------------------------
% ---------------------------------------------------------------------
% Configuración general

\usepackage[utf8]{inputenc}
\usepackage[T1]{fontenc}

\ifcastellano\usepackage[spanish,es-tabla]{babel}\fi
\ifvalencia\usepackage[catalan]{babel}\fi
\ifenglish\usepackage[english]{babel}\usepackage{csquotes}\fi
\newcommand{\sen}{\on{sen}}	

\usepackage{xcolor}
\usepackage{array,booktabs}
\usepackage{tabularx,longtable,multicol,multirow}
\usepackage{subcaption}
\usepackage{amsmath}
\usepackage{amssymb}

%\ifenglish
%	\raggedright % No justifica, i no divide las palabras con guiones
%\fi

% ---------------------------------------------------------------------
% ---------------------------------------------------------------------
% Bibliografía

\usepackage[
	url = false,
	style = numeric,
	hyperref = true,
	backref = true,
	backend = biber, % Otra opción es 'backend = bibtex'
	%backend = bibtex,
	]{biblatex}

% ---------------------------------------------------------------------
% ---------------------------------------------------------------------
% Documento electrónico

\usepackage{makeidx}
\makeindex

\ifEBOOKPDF
	\colorlet{colorEnlace}{red!75!black} 
\else
    \ifImprimir
	    \colorlet{colorEnlace}{black} 
    \else
        \colorlet{colorEnlace}{blue!85!black} 
    \fi
\fi

\usepackage[
	{colorlinks},
	{linkcolor=colorEnlace},
	{citecolor=colorEnlace},
	{urlcolor=colorEnlace},
	{bookmarksnumbered},
	{breaklinks},
	]{hyperref}

% ---------------------------------------------------------------------
% ---------------------------------------------------------------------
% Expresión de unidades según el Sistema Internacional, monedas

\usepackage{eurosym}
\usepackage{siunitx}

\ifenglish
	\sisetup{output-decimal-marker={.}}
\else
	\sisetup{output-decimal-marker={,}}
\fi

\DeclareSIUnit[number-unit-product = {\;}] \EURO{\geneuro}

% ---------------------------------------------------------------------

\newcommand{\ingles}[1]{\textit{#1}}
\newenvironment{ParrafoIngles}{\itshape}{}

% ------------------------------------------------------------------------

\usepackage{xspace}

\newcommand{\angles}[1]{\textit{#1}\/}
\newcommand{\miUrl}[1]{{\small%
	%\texttt%
	{\underline{#1}}}}

\newcommand{\matlabr}{{\sc Matlab}$^\circledR$\xspace}
\newcommand{\simulinkr}{\textit{Simulink}$^\circledR$\xspace}
\newcommand{\matlab}{{\textsc{Matlab}}\xspace}
\newcommand{\simulink}{\textit{Simulink}\xspace}

\newcommand{\scr}{\textit{script\/}\xspace}
\newcommand{\scrs}{\textit{scripts\/}\xspace}

% ------------------------------------------------------------------------

\definecolor{griset}{rgb}{.925, .925, .925}

\newsavebox{\mybox}
\newenvironment{parrafoDestacado}
	{%
	\fboxsep = 2ex
	\fboxrule = .4pt
  	\begin{lrbox}{\mybox}%
  	\begin{minipage}{.85\textwidth-2\fboxsep}\itshape\parskip=2ex
	}
	{%
	\end{minipage}
  	\end{lrbox}%
	\begin{flushright}
		\colorbox{griset}{\usebox{\mybox}}%
  		%\fcolorbox{black}{griset}{\usebox{\mybox}}%
	\end{flushright}
	}

% ------------------------------------------------------------------------
% Resumen del capítulo

\newsavebox{\myboxb}
\newenvironment{Resumen}
	{%
	\vspace*{-2.0cm}
	\fboxsep = 0pt
	\fboxrule = 0pt
  	\begin{lrbox}{\myboxb}%
  	\begin{minipage}{.85\textwidth}\itshape\parskip=2ex\parindent=2em
	}
	{%
	\end{minipage}
  	\end{lrbox}%
	\begin{flushright}
		\usebox{\myboxb}%
	\end{flushright}
	\vspace{0.5cm}
	}

% ---------------------------------------------------------------------
% ---------------------------------------------------------------------
% Símbolos matemáticos

\newcommand{\on}{\operatorname}
\usepackage{amsmath}
\DeclareMathOperator*{\argmax}{argmax}
\DeclareMathOperator*{\argmin}{argmin}

% ---------------------------------------------------------------------
% ---------------------------------------------------------------------
% Teoremas y ejemplos

\ifcastellano
	\newtheorem{teorema}{\upshape\bfseries Teorema}[section]
	\newtheorem{lema}{\mdseries\scshape Lema}[section]
	\newtheorem{proposicion}{\upshape\bfseries Proposición}[section]
	\newtheorem{ejemplo}{\bfseries\scshape Ejemplo}[section]
\fi

\ifvalencia % Es mantenen els mateixos noms per compatibilitat, però l'autor els pot personalitzar
	\newtheorem{teorema}{\upshape\bfseries Teorema}[section]
	\newtheorem{lema}{\mdseries\scshape Lema}[section]
	\newtheorem{proposicion}{\upshape\bfseries Proposició}[section]
	\newtheorem{ejemplo}{\bfseries\scshape Exemple}[section]
\fi

\ifenglish
	\newtheorem{teorema}{\upshape\bfseries Theorem}[section]
	\newtheorem{lema}{\mdseries\scshape Lemma}[section]
	\newtheorem{proposicion}{\upshape\bfseries Proposition}[section]
	\newtheorem{ejemplo}{\bfseries\scshape Example}[section]
\fi

% ---------------------------------------------------------------------

